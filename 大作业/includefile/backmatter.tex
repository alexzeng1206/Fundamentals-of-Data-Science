% !Mode:: "TeX:UTF-8"
%%%%%%%%%%%%%%%%%%%%%%%%%%%%-------结论--------%%%%%%%%%%%%%%%%%%%%%%%%%%%%%%%%

\acknowledgement
\addcontentsline{toc}{chapter}{结论}
%\linespread{1.5}

概率模型有时既含有观测变量,又含有隐变量。如果概率模型的变量都是观测变量,那么给定数据,可以直接用极大似然估计法,或贝叶斯估计法估计模型参数。但是,当模型含有隐变量时,就不能简单地使用这些估计方法。EM算法就是含有隐变量的概率模型参数的极大似然估计法或极大后验概率估计法。高斯混合模型是应用广泛的聚类算法,也是生成模型。GMM是对高斯模型进行简单的扩展,它使用多个高斯分布的组合来刻画数据分布。在许多情况下,EM算法是学习GMM的有效方法。

% 这里写本次实验的结论
















